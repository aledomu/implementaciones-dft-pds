\documentclass{beamer}
\usepackage[utf8]{inputenc}
\usepackage[spanish]{babel}
\usepackage{tikz}
\usepackage{biblatex}
\usepackage{csquotes}

\addbibresource{references.bib}
\nocite{*}

\title{Otras implementaciones de la DFT}
\author{Alejandro Domínguez Muñoz}
\institute{Universidad de Sevilla}
\date{Noviembre de 2022}

\begin{document}

\frame{\titlepage}

\frame{\tableofcontents}

\section{Introducción}

\begin{frame}{Recordatorio}
\begin{gather*}
\mathbb{R}^{(N : \mathbb{N})} \equiv (N : \mathbb{N}) \rightarrow N \rightarrow \mathbb{R} \\
x : \mathbb{R}^{(N : \mathbb{N})} \\
k : \mathbb{N} \\
k \in [0, N - 1] \\
W_N = e^{-j2\pi/N} \\
X(k) = \sum_{n = 0}^{N - 1} x(n) W_N^{kn}
\end{gather*}
\end{frame}

\section{Divide y vencerás}

\begin{frame}{Subdivisión de la DFT}
\begin{itemize}
\item $N = L \cdot M$ (rellenar con 0s si no cuadra)
\item Representar como $\mathbb{R}^{(N : \mathbb{N})}$ o $\mathbb{R}^{(L \times M : \mathbb{N} \times \mathbb{N})}$
\item Siguiendo las correspondencias entre índices:

\begin{align*}
\only<1>{
X(p, q) &= \sum_{m = 0}^{M - 1} \sum_{l = 0}^{L - 1} x'(l, m) W_N^{(Mp + q)(mL + l)} \\
X(p, q) &= \sum_{l = 0}^{L - 1} W_N^{lq} \left[\sum_{m = 0}^{M - 1} x'(l, m) W_M^{mq}\right] W_L^{lp} \\
X(p, q) &= \sum_{l = 0}^{L - 1} W_N^{lq} \left[\sum_{m = 0}^{M - 1} x(Ml + m) W_M^{mq}\right] W_L^{lp}
}
\only<2>{
X(p, q) &= \sum_{m = 0}^{M - 1} \sum_{l = 0}^{L - 1} x'(l, m) W_N^{mp} W_L^{lp} W_M^{mq} \\
X(p, q) &= \sum_{m = 0}^{M - 1} W_M^{mq} \left[\sum_{l = 0}^{L - 1} x'(l, m) W_M^{lp}\right] W_L^{mp} \\
X(p, q) &= \sum_{m = 0}^{M - 1} W_M^{mq} \left[\sum_{l = 0}^{L - 1} x(l + mL) W_M^{lp}\right] W_L^{mp}
}
\end{align*}

\end{itemize}
\end{frame}

\section{Número de muestras de base 2}

\begin{frame}{$N = 2^k$: Diezmado temporal}
\begin{equation*}
\begin{split}
\only<1>{
X(k) &= \sum_{n = 0}^{N - 1} x(n) W_N^{kn} \\
     &= \sum_{n = 0}^{(N/2) - 1} x(2n) W_N^{2nk}
        + \sum_{n = 0}^{(N/2) - 1} x(2n + 1) W_N^{k(2n + 1)} \\
     &= \sum_{n = 0}^{(N/2) - 1} x(2n) W_{N/2}^{kn}
        + W_{N}^{k} \sum_{n = 0}^{(N/2) - 1} x(2n + 1) W_{N/2}^{kn}
}
\only<2>{
X(k') &= \sum_{n = 0}^{(N/2) - 1} x(2n) W_{N/2}^{k'n}
         + W_{N}^{k'} \sum_{n = 0}^{(N/2) - 1} x(2n + 1) W_{N/2}^{k'n} \\
X(k' + \frac{N}{2}) &= \sum_{n = 0}^{(N/2) - 1} x(2n) W_{N/2}^{k'n}
                       - W_{N}^{k'} \sum_{n = 0}^{(N/2) - 1} x(2n + 1) W_{N/2}^{k'n}
}
\end{split}
\end{equation*}
\end{frame}

\begin{frame}{$N = 2^k$: Diezmado de frecuencia}
\begin{equation*}
\begin{split}
\only<1>{
X(k) &= \sum_{n = 0}^{N - 1} x(n) W_N^{kn} \\
     &= \sum_{n = 0}^{(N/2) - 1} x(n) W_N^{kn}
        + \sum_{n = N/2}^{N - 1} x(n + \frac{N}{2}) W_N^{kn} \\
     &= \sum_{n = 0}^{(N/2) - 1} x(n) W_N^{kn}
        + W_N^{kN/2} \sum_{n = 0}^{(N/2) - 1} x(n + \frac{N}{2}) W_N^{kn} \\
     &= \sum_{n = 0}^{(N/2) - 1} \left[ x(n) + (-1)^k x(n + \frac{N}{2}) \right] W_N^{kn}
}
\only<2>{
X(2k') &= \sum_{n = 0}^{(N/2) - 1} \left[ x(n) + x(n + \frac{N}{2}) \right] W_{N/2}^{k'n} \\
X(2k' + 1) &= \sum_{n = 0}^{(N/2) - 1} \left[ x(n) + x(n + \frac{N}{2}) \right] W_{N}^{n} W_{N/2}^{k'n}
}
\end{split}
\end{equation*}
\end{frame}

\section{Número de muestras de base 4}

\begin{frame}{$N = 4^k$: Diezmado temporal}
Aplicamos la expresión obtenida para el algoritmo divide y vencerás,
con $L = 4$ (y por lo tanto $M = \frac{N}{4}$):

\begin{align*}
X(q + \frac{N}{4}p) &= \sum_{l = 0}^3 W_N^{lq} \left[\sum_{m = 0}^{(N/4) - 1} x(4m + l) W_{N/4}^{mq}\right] W_4^{lp} \\
\end{align*}
\end{frame}

\begin{frame}{$N = 4^k$: Diezmado de frecuencia}
Aplicamos la expresión obtenida para el algoritmo divide y vencerás,
con $M = 4$ (y por lo tanto $L = \frac{N}{4}$):

\begin{align*}
X(4p + q) &= \sum_{l = 0}^{(N/4) - 1} W_{N}^{lq} \left[\sum_{m = 0}^3 x(4m + l) W_{4}^{mq}\right] W_{N/4}^{lp} \\
\end{align*}
\end{frame}

\section{Número de muestras de base dividida}

\begin{frame}{$N = 2^k$: Split-radix}
\only<1>{
\begin{itemize}
\item Algoritmo de diezmado \textbf{en frecuencia}.
\item Para muestras pares, $X(2k')$.
\item Para muestras impares, $X(4k'' + 1)$ y $X(4k'' + 3)$.
\end{itemize}
}
\only<2>{
\begin{equation*}
X(2k') = \sum_{n = 0}^{(N/2) - 1} \left[ x(n) + x(n + \frac{N}{2}) \right] W_{N/2}^{k'n}
\end{equation*}
\begin{multline*}
X(4k'' + 1) = \sum_{n = 0}^{(N/4) - 1} \left[ x(n) - x(n + \frac{N}{2}) \right] \\
              - j \left[ x(n + \frac{N}{4}) - x(n + \frac{3N}{4}) \right] W_{N}^{n} W_{N/4}^{k''n}
\end{multline*}
\begin{multline*}
X(4k'' + 3) = \sum_{n = 0}^{(N/4) - 1} \left[ x(n) - x(n + \frac{N}{2}) \right] \\
              + j \left[ x(n + \frac{N}{4}) - x(n + \frac{3N}{4}) \right] W_{N}^{3n} W_{N/4}^{k''n}
\end{multline*}
}
\end{frame}

\begin{frame}{Bibliografía}
\printbibliography
\end{frame}


\end{document}